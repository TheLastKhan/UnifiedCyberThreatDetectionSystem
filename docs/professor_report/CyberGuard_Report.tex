% Options for packages loaded elsewhere
\PassOptionsToPackage{unicode}{hyperref}
\PassOptionsToPackage{hyphens}{url}
\documentclass[
]{article}
\usepackage{xcolor}
\usepackage{amsmath,amssymb}
\setcounter{secnumdepth}{-\maxdimen} % remove section numbering
\usepackage{iftex}
\ifPDFTeX
  \usepackage[T1]{fontenc}
  \usepackage[utf8]{inputenc}
  \usepackage{textcomp} % provide euro and other symbols
\else % if luatex or xetex
  \usepackage{unicode-math} % this also loads fontspec
  \defaultfontfeatures{Scale=MatchLowercase}
  \defaultfontfeatures[\rmfamily]{Ligatures=TeX,Scale=1}
\fi
\usepackage{lmodern}
\ifPDFTeX\else
  % xetex/luatex font selection
\fi
% Use upquote if available, for straight quotes in verbatim environments
\IfFileExists{upquote.sty}{\usepackage{upquote}}{}
\IfFileExists{microtype.sty}{% use microtype if available
  \usepackage[]{microtype}
  \UseMicrotypeSet[protrusion]{basicmath} % disable protrusion for tt fonts
}{}
\makeatletter
\@ifundefined{KOMAClassName}{% if non-KOMA class
  \IfFileExists{parskip.sty}{%
    \usepackage{parskip}
  }{% else
    \setlength{\parindent}{0pt}
    \setlength{\parskip}{6pt plus 2pt minus 1pt}}
}{% if KOMA class
  \KOMAoptions{parskip=half}}
\makeatother
\usepackage{color}
\usepackage{fancyvrb}
\newcommand{\VerbBar}{|}
\newcommand{\VERB}{\Verb[commandchars=\\\{\}]}
\DefineVerbatimEnvironment{Highlighting}{Verbatim}{commandchars=\\\{\}}
% Add ',fontsize=\small' for more characters per line
\newenvironment{Shaded}{}{}
\newcommand{\AlertTok}[1]{\textcolor[rgb]{1.00,0.00,0.00}{\textbf{#1}}}
\newcommand{\AnnotationTok}[1]{\textcolor[rgb]{0.38,0.63,0.69}{\textbf{\textit{#1}}}}
\newcommand{\AttributeTok}[1]{\textcolor[rgb]{0.49,0.56,0.16}{#1}}
\newcommand{\BaseNTok}[1]{\textcolor[rgb]{0.25,0.63,0.44}{#1}}
\newcommand{\BuiltInTok}[1]{\textcolor[rgb]{0.00,0.50,0.00}{#1}}
\newcommand{\CharTok}[1]{\textcolor[rgb]{0.25,0.44,0.63}{#1}}
\newcommand{\CommentTok}[1]{\textcolor[rgb]{0.38,0.63,0.69}{\textit{#1}}}
\newcommand{\CommentVarTok}[1]{\textcolor[rgb]{0.38,0.63,0.69}{\textbf{\textit{#1}}}}
\newcommand{\ConstantTok}[1]{\textcolor[rgb]{0.53,0.00,0.00}{#1}}
\newcommand{\ControlFlowTok}[1]{\textcolor[rgb]{0.00,0.44,0.13}{\textbf{#1}}}
\newcommand{\DataTypeTok}[1]{\textcolor[rgb]{0.56,0.13,0.00}{#1}}
\newcommand{\DecValTok}[1]{\textcolor[rgb]{0.25,0.63,0.44}{#1}}
\newcommand{\DocumentationTok}[1]{\textcolor[rgb]{0.73,0.13,0.13}{\textit{#1}}}
\newcommand{\ErrorTok}[1]{\textcolor[rgb]{1.00,0.00,0.00}{\textbf{#1}}}
\newcommand{\ExtensionTok}[1]{#1}
\newcommand{\FloatTok}[1]{\textcolor[rgb]{0.25,0.63,0.44}{#1}}
\newcommand{\FunctionTok}[1]{\textcolor[rgb]{0.02,0.16,0.49}{#1}}
\newcommand{\ImportTok}[1]{\textcolor[rgb]{0.00,0.50,0.00}{\textbf{#1}}}
\newcommand{\InformationTok}[1]{\textcolor[rgb]{0.38,0.63,0.69}{\textbf{\textit{#1}}}}
\newcommand{\KeywordTok}[1]{\textcolor[rgb]{0.00,0.44,0.13}{\textbf{#1}}}
\newcommand{\NormalTok}[1]{#1}
\newcommand{\OperatorTok}[1]{\textcolor[rgb]{0.40,0.40,0.40}{#1}}
\newcommand{\OtherTok}[1]{\textcolor[rgb]{0.00,0.44,0.13}{#1}}
\newcommand{\PreprocessorTok}[1]{\textcolor[rgb]{0.74,0.48,0.00}{#1}}
\newcommand{\RegionMarkerTok}[1]{#1}
\newcommand{\SpecialCharTok}[1]{\textcolor[rgb]{0.25,0.44,0.63}{#1}}
\newcommand{\SpecialStringTok}[1]{\textcolor[rgb]{0.73,0.40,0.53}{#1}}
\newcommand{\StringTok}[1]{\textcolor[rgb]{0.25,0.44,0.63}{#1}}
\newcommand{\VariableTok}[1]{\textcolor[rgb]{0.10,0.09,0.49}{#1}}
\newcommand{\VerbatimStringTok}[1]{\textcolor[rgb]{0.25,0.44,0.63}{#1}}
\newcommand{\WarningTok}[1]{\textcolor[rgb]{0.38,0.63,0.69}{\textbf{\textit{#1}}}}
\usepackage{longtable,booktabs,array}
\newcounter{none} % for unnumbered tables
\usepackage{calc} % for calculating minipage widths
% Correct order of tables after \paragraph or \subparagraph
\usepackage{etoolbox}
\makeatletter
\patchcmd\longtable{\par}{\if@noskipsec\mbox{}\fi\par}{}{}
\makeatother
% Allow footnotes in longtable head/foot
\IfFileExists{footnotehyper.sty}{\usepackage{footnotehyper}}{\usepackage{footnote}}
\makesavenoteenv{longtable}
\setlength{\emergencystretch}{3em} % prevent overfull lines
\providecommand{\tightlist}{%
  \setlength{\itemsep}{0pt}\setlength{\parskip}{0pt}}
\usepackage{bookmark}
\IfFileExists{xurl.sty}{\usepackage{xurl}}{} % add URL line breaks if available
\urlstyle{same}
\hypersetup{
  hidelinks,
  pdfcreator={LaTeX via pandoc}}

\author{}
\date{}

\begin{document}

\section{CyberGuard}\label{cyberguard}

\subsection{Birleşik Siber Tehdit Tespit
Sistemi}\label{birleux15fik-siber-tehdit-tespit-sistemi}

\subsubsection{Proje Final Raporu}\label{proje-final-raporu}

\begin{center}\rule{0.5\linewidth}{0.5pt}\end{center}

\textbf{Versiyon:} 2.0.0\\
\textbf{Tarih:} Ocak 2026

\begin{center}\rule{0.5\linewidth}{0.5pt}\end{center}

\subsection{İÇİNDEKİLER}\label{iuxe7indekiler}

{\def\LTcaptype{none} % do not increment counter
\begin{longtable}[]{@{}ll@{}}
\toprule\noalign{}
Bölüm & Sayfa \\
\midrule\noalign{}
\endhead
\bottomrule\noalign{}
\endlastfoot
1. YÖNETİCİ ÖZETİ & 3 \\
2. SİSTEM GENEL BAKIŞ & 4 \\
2.1 Amaç ve Hedefler & 4 \\
2.2 Kapsam & 4 \\
2.3 Teknoloji Yığını & 5 \\
3. YAZILIM MİMARİSİ VE TASARIM & 6 \\
3.1 Mimari Karakterizasyon & 6 \\
3.2 Mimari Kararların Gerekçeleri & 7 \\
3.3 Katman Ayrımı ve Sorumluluklar & 8 \\
4. MİMARİ KALIPLAR VE TASARIM DESENLERİ & 9 \\
4.1 Pattern-Mapping Tablosu & 9 \\
4.2 Kalıp Seçim Gerekçeleri & 10 \\
5. SİSTEM ÖZELLİKLERİ VE KULLANICI ARAYÜZÜ & 11 \\
5.1 Ana Panel (Dashboard) & 11 \\
5.2 E-posta Analizi & 12 \\
5.3 Web Log Analizi & 13 \\
5.4 Korelasyon Analizi & 14 \\
5.5 Model Karşılaştırma & 15 \\
5.6 Raporlar ve Ayarlar & 16 \\
6. YAPAY ZEKA MODELLERİ & 17 \\
6.1 BERT (DistilBERT) & 17 \\
6.2 FastText & 17 \\
6.3 TF-IDF + Random Forest & 18 \\
6.4 Isolation Forest (Web Analizi) & 18 \\
7. TEST METODOLOJİSİ VE SONUÇLARI & 19 \\
7.1 Test Stratejisi ve Amacı & 19 \\
7.2 Neden Bu Testler Seçildi? & 20 \\
7.3 Fonksiyonel Test Sonuçları & 21 \\
7.4 Performans Metrikleri & 21 \\
8. MODEL KARŞILAŞTIRMASI VE TRADE-OFF ANALİZİ & 22 \\
8.1 Performans Karşılaştırması & 22 \\
8.2 BERT Neden Daha İyi Performans Gösterdi? & 23 \\
8.3 Hız vs Doğruluk Trade-off & 24 \\
8.4 False Positive / False Negative Analizi & 25 \\
8.5 Concept Drift Riski ve Çözüm Stratejileri & 26 \\
9. API REFERANSI & 27 \\
10. KURULUM VE YAPILANDIRMA & 28 \\
11. SONUÇ & 29 \\
\end{longtable}
}

\begin{center}\rule{0.5\linewidth}{0.5pt}\end{center}

\subsection{1. YÖNETİCİ ÖZETİ}\label{yuxf6netici-uxf6zeti}

CyberGuard, kurumsal siber güvenlik ihtiyaçlarına yönelik geliştirilmiş,
yapay zeka destekli bir tehdit tespit platformudur. Sistem, e-posta
tabanlı phishing saldırıları ile web tabanlı saldırıları (SQL Injection,
XSS, DDoS) gerçek zamanlı olarak tespit etme kapasitesine sahiptir.

\subsubsection{Temel Özellikler}\label{temel-uxf6zellikler}

{\def\LTcaptype{none} % do not increment counter
\begin{longtable}[]{@{}
  >{\raggedright\arraybackslash}p{(\linewidth - 4\tabcolsep) * \real{0.3462}}
  >{\raggedright\arraybackslash}p{(\linewidth - 4\tabcolsep) * \real{0.3846}}
  >{\raggedright\arraybackslash}p{(\linewidth - 4\tabcolsep) * \real{0.2692}}@{}}
\toprule\noalign{}
\begin{minipage}[b]{\linewidth}\raggedright
Özellik
\end{minipage} & \begin{minipage}[b]{\linewidth}\raggedright
Açıklama
\end{minipage} & \begin{minipage}[b]{\linewidth}\raggedright
Durum
\end{minipage} \\
\midrule\noalign{}
\endhead
\bottomrule\noalign{}
\endlastfoot
E-posta Phishing Tespiti & 3 farklı AI modeli (BERT, FastText, TF-IDF)
ile ensemble analiz & ✅ Çalışıyor \\
Web Log Analizi & Isolation Forest ile SQL Injection, XSS, DDoS tespiti
& ✅ Çalışıyor \\
Korelasyon Analizi & E-posta ve web tehditlerinin zaman/IP bazlı
ilişkilendirilmesi & ✅ Çalışıyor \\
Gerçek Zamanlı Dashboard & Chart.js ile interaktif grafikler ve anlık
istatistikler & ✅ Çalışıyor \\
Çoklu Dil Desteği & Türkçe ve İngilizce kullanıcı arayüzü & ✅
Çalışıyor \\
Docker Deployment & 6 container ile hazır production dağıtımı & ✅
Çalışıyor \\
REST API & 15+ endpoint ile tam entegrasyon imkanı & ✅ Çalışıyor \\
\end{longtable}
}

\begin{center}\rule{0.5\linewidth}{0.5pt}\end{center}

\subsection{2. SİSTEM GENEL BAKIŞ}\label{sistem-genel-bakiux15f}

\subsubsection{2.1 Amaç ve Hedefler}\label{amauxe7-ve-hedefler}

\textbf{Birincil Amaç:} Kurumsal ortamlarda e-posta ve web tabanlı siber
tehditleri yapay zeka teknolojileri kullanarak otomatik olarak tespit
etmek ve raporlamak.

\textbf{Hedefler:} - Phishing e-postalarını \%90+ doğrulukla tespit
etmek - Web saldırı girişimlerini gerçek zamanlı olarak belirlemek -
Farklı vektörlerden gelen tehditleri ilişkilendirmek (korelasyon) -
Güvenlik analistlerine kullanımı kolay bir arayüz sunmak - Mevcut
güvenlik altyapılarına API üzerinden entegre olmak

\subsubsection{2.2 Kapsam}\label{kapsam}

{\def\LTcaptype{none} % do not increment counter
\begin{longtable}[]{@{}ll@{}}
\toprule\noalign{}
Kapsam İçi & Kapsam Dışı \\
\midrule\noalign{}
\endhead
\bottomrule\noalign{}
\endlastfoot
E-posta phishing tespiti & Ağ trafiği analizi \\
Web log anomali analizi & Endpoint koruma \\
Tehdit korelasyonu & Malware analizi \\
Raporlama ve export & Otomatik müdahale \\
\end{longtable}
}

\subsubsection{2.3 Teknoloji
Yığını}\label{teknoloji-yux131ux11fux131nux131}

{\def\LTcaptype{none} % do not increment counter
\begin{longtable}[]{@{}lll@{}}
\toprule\noalign{}
Katman & Teknoloji & Versiyon \\
\midrule\noalign{}
\endhead
\bottomrule\noalign{}
\endlastfoot
Backend & Python, Flask, Gunicorn & 3.8+, 2.0+, 21.0+ \\
Frontend & HTML5, CSS3, JavaScript, Chart.js & ES6+, 4.0+ \\
Veritabanı & PostgreSQL, SQLAlchemy & 15.0, 2.0+ \\
Önbellek & Redis & 7.0+ \\
AI/ML & scikit-learn, PyTorch, Transformers & 1.0+, 2.0+, 4.0+ \\
NLP & NLTK, spaCy, FastText & 3.8+, 3.0+, - \\
Konteynerizasyon & Docker, Docker Compose & 24.0+, 2.0+ \\
İzleme & Prometheus, Grafana & 2.45+, 10.0+ \\
\end{longtable}
}

\begin{center}\rule{0.5\linewidth}{0.5pt}\end{center}

\subsection{3. YAZILIM MİMARİSİ VE
TASARIM}\label{yazilim-mimarisi-ve-tasarim}

\subsubsection{3.1 Mimari Karakterizasyon}\label{mimari-karakterizasyon}

CyberGuard, \textbf{modüler, servis-odaklı bir mimari} üzerine inşa
edilmiştir. Sistemin mimari karakteri şu şekilde tanımlanabilir:

\begin{quote}
\emph{``CyberGuard is designed as a modular, service-oriented
architecture where the sensing logic and presentation layers are
separated, which allows machine learning models to develop
independently.''}
\end{quote}

\paragraph{Mimari Tipi: Request-Response + Event-Driven
Hybrid}\label{mimari-tipi-request-response-event-driven-hybrid}

Sistem temel olarak \textbf{request-response} paradigmasını kullanmakla
birlikte, tehdit tespiti ve korelasyon analizi bileşenlerinde
\textbf{event-driven} yaklaşımı benimser:

{\def\LTcaptype{none} % do not increment counter
\begin{longtable}[]{@{}
  >{\raggedright\arraybackslash}p{(\linewidth - 4\tabcolsep) * \real{0.3000}}
  >{\raggedright\arraybackslash}p{(\linewidth - 4\tabcolsep) * \real{0.3667}}
  >{\raggedright\arraybackslash}p{(\linewidth - 4\tabcolsep) * \real{0.3333}}@{}}
\toprule\noalign{}
\begin{minipage}[b]{\linewidth}\raggedright
Bileşen
\end{minipage} & \begin{minipage}[b]{\linewidth}\raggedright
Paradigma
\end{minipage} & \begin{minipage}[b]{\linewidth}\raggedright
Açıklama
\end{minipage} \\
\midrule\noalign{}
\endhead
\bottomrule\noalign{}
\endlastfoot
Dashboard → API & Request-Response & Kullanıcı istekleri synchronous
olarak işlenir \\
Email/Web Log → Detection & Event-Driven & Gelen veriler event olarak
işlenir \\
Detection → Correlation & Publisher-Subscriber & Tehditler korelasyon
motoruna publish edilir \\
Correlation → Alerts & Event-Driven & Koordineli saldırılarda alert
event'leri oluşur \\
\end{longtable}
}

\paragraph{Mimari Diyagram}\label{mimari-diyagram}

\begin{verbatim}
┌─────────────────────────────────────────────────────────────────────────┐
│                         KULLANICI ARAYÜZÜ                               │
│  ┌─────────────┐ ┌─────────────┐ ┌─────────────┐ ┌─────────────────┐   │
│  │  Dashboard  │ │   Email     │ │   Web Log   │ │    Raporlar     │   │
│  │   Paneli    │ │   Analizi   │ │   Analizi   │ │   & Ayarlar     │   │
│  └──────┬──────┘ └──────┬──────┘ └──────┬──────┘ └────────┬────────┘   │
└─────────┼───────────────┼───────────────┼─────────────────┼────────────┘
          │               │               │                 │
          └───────────────┼───────────────┼─────────────────┘
                          ▼               ▼
┌─────────────────────────────────────────────────────────────────────────┐
│                          FLASK REST API                                  │
│  /api/email/*  │  /api/predict/*  │  /api/correlation/*  │  /api/*     │
└─────────────────────────────────────────────────────────────────────────┘
                          │               │
          ┌───────────────┼───────────────┼───────────────┐
          ▼               ▼               ▼               ▼
┌─────────────┐  ┌─────────────┐  ┌─────────────┐  ┌─────────────┐
│    BERT     │  │  FastText   │  │  TF-IDF+RF  │  │  Isolation  │
│ (DistilBERT)│  │   Model     │  │   Model     │  │   Forest    │
└─────────────┘  └─────────────┘  └─────────────┘  └─────────────┘
                          │               │
                          ▼               ▼
┌─────────────────────────────────────────────────────────────────────────┐
│                          VERİ KATMANI                                    │
│  ┌────────────┐  ┌────────────┐  ┌────────────┐  ┌────────────┐        │
│  │ PostgreSQL │  │   Redis    │  │ Prometheus │  │  Grafana   │        │
│  │ (Veritabanı)│ │  (Cache)   │  │ (Metrikler)│  │ (Dashboard)│        │
│  └────────────┘  └────────────┘  └────────────┘  └────────────┘        │
└─────────────────────────────────────────────────────────────────────────┘
\end{verbatim}

\subsubsection{3.2 Mimari Kararların
Gerekçeleri}\label{mimari-kararlarux131n-gerekuxe7eleri}

\paragraph{Neden Phishing ve Web Log Aynı
Backend'de?}\label{neden-phishing-ve-web-log-aynux131-backendde}

\textbf{Karar:} E-posta phishing tespiti ve web log analizi tek bir
Flask API backend'inde birleştirilmiştir.

\textbf{Gerekçe:} - \textbf{Korelasyon Avantajı:} Aynı IP adresinden
gelen phishing e-postası ve web saldırısı, paylaşımlı veri katmanı
sayesinde hızlıca ilişkilendirilebilir - \textbf{Kaynak Verimliliği:}
Tek container, düşük memory footprint (küçük/orta ölçekli kurumlar için
ideal) - \textbf{Deployment Basitliği:} Tek docker image, kolay bakım ve
güncelleme - \textbf{Veri Tutarlılığı:} Merkezi PostgreSQL veritabanı,
tüm tehdit verileri için single source of truth

\textbf{Alternatif Değerlendirme:} Microservice mimarisine geçiş, yüksek
ölçeklenebilirlik için düşünülebilir ancak mevcut kullanım senaryosu
için overengineering olarak değerlendirilmiştir.

\paragraph{Neden Model Inference API
İçinde?}\label{neden-model-inference-api-iuxe7inde}

\textbf{Karar:} ML modelleri (BERT, FastText, TF-IDF) doğrudan Flask API
container'ı içinde çalıştırılmaktadır.

\textbf{Gerekçe:} - \textbf{Latency Optimizasyonu:} Model → API arası
network hop'u elimine edilmiştir (\textasciitilde5-10ms tasarruf) -
\textbf{Session State:} Modeller bir kez yüklenir ve memory'de tutulur
(cold start yok) - \textbf{Debugging Kolaylığı:} End-to-end tracing tek
process'te yapılabilir - \textbf{Resource Isolation:} Docker container
zaten izolasyon sağlar

\textbf{Trade-off:} Bu yaklaşım horizontal scaling'i zorlaştırır. Yüksek
throughput senaryolarında TensorFlow Serving veya TorchServe gibi
dedicated inference server'lara geçiş önerilir.

\subsubsection{3.3 Katman Ayrımı ve
Sorumluluklar}\label{katman-ayrux131mux131-ve-sorumluluklar}

{\def\LTcaptype{none} % do not increment counter
\begin{longtable}[]{@{}
  >{\raggedright\arraybackslash}p{(\linewidth - 4\tabcolsep) * \real{0.2581}}
  >{\raggedright\arraybackslash}p{(\linewidth - 4\tabcolsep) * \real{0.3548}}
  >{\raggedright\arraybackslash}p{(\linewidth - 4\tabcolsep) * \real{0.3871}}@{}}
\toprule\noalign{}
\begin{minipage}[b]{\linewidth}\raggedright
Katman
\end{minipage} & \begin{minipage}[b]{\linewidth}\raggedright
Teknoloji
\end{minipage} & \begin{minipage}[b]{\linewidth}\raggedright
Sorumluluk
\end{minipage} \\
\midrule\noalign{}
\endhead
\bottomrule\noalign{}
\endlastfoot
\textbf{Presentation Layer (View)} & Flask Dashboard + Jinja2 +
JavaScript & Kullanıcı etkileşimi, form handling, data visualization \\
\textbf{Application Layer (Controller)} & Flask REST API Routes &
Business logic orchestration, input sanitization, response formatting \\
\textbf{Domain Layer (Model)} & Email Detector, Web Analyzer,
Correlation Engine & ML inference, feature extraction, risk scoring \\
\textbf{Data Layer (Persistence)} & PostgreSQL + Redis + File System &
Data persistence, caching, model storage \\
\end{longtable}
}

\begin{center}\rule{0.5\linewidth}{0.5pt}\end{center}

\subsection{4. MİMARİ KALIPLAR VE TASARIM
DESENLERİ}\label{mimari-kaliplar-ve-tasarim-desenleri}

CyberGuard sistemi, bilinen birçok mimari ve tasarım modelini örtük
olarak benimser. Sistem açıkça tek bir model etrafında tasarlanmamış
olsa da, modüler yapısı doğal olarak MVC ve olay odaklı prensiplerle
uyumludur. Bu yaklaşım, sistemin \textbf{bakım kolaylığını},
\textbf{ölçeklenebilirliğini} ve \textbf{genişletilebilirliğini}
artırır.

\subsubsection{4.1 Pattern-Mapping
Tablosu}\label{pattern-mapping-tablosu}

{\def\LTcaptype{none} % do not increment counter
\begin{longtable}[]{@{}
  >{\raggedright\arraybackslash}p{(\linewidth - 2\tabcolsep) * \real{0.5345}}
  >{\raggedright\arraybackslash}p{(\linewidth - 2\tabcolsep) * \real{0.4655}}@{}}
\toprule\noalign{}
\begin{minipage}[b]{\linewidth}\raggedright
Mimari Kalıp / Tasarım Deseni
\end{minipage} & \begin{minipage}[b]{\linewidth}\raggedright
CyberGuard'daki Karşılığı
\end{minipage} \\
\midrule\noalign{}
\endhead
\bottomrule\noalign{}
\endlastfoot
\textbf{Model-View-Controller (MVC)} & Dashboard (View), Flask API
(Controller), PostgreSQL + ML Models (Model) \\
\textbf{Event-Driven / Publisher-Subscriber} & Email/Web log ingestion →
Detection → Correlation → Alert zinciri \\
\textbf{Ensemble Learning Pattern} & BERT, FastText ve TF-IDF
sonuçlarının weighted voting ile birleştirilmesi (ağırlıklar: 0.5, 0.3,
0.2) \\
\textbf{Cache-Aside Pattern} & Redis ile sık erişilen dashboard
istatistiklerinin cachelenmesi (TTL: 60s) \\
\textbf{Repository Pattern} & SQLAlchemy ORM ile database abstraction \\
\textbf{Factory Pattern} & \texttt{get\_bert\_detector()},
\texttt{get\_fasttext\_detector()} singleton-like instance'lar \\
\textbf{Strategy Pattern} & Tüm detectorlar \texttt{predict()} ve
\texttt{predict\_with\_explanation()} metodlarını implement eder \\
\textbf{Façade Pattern} & \texttt{/api/email/analyze/hybrid} endpoint'i
3 modeli tek interface arkasında gizler \\
\textbf{Circuit Breaker Pattern} & VirusTotal API erişilemezse ML-based
detection ile devam \\
\end{longtable}
}

\subsubsection{4.2 Kalıp Seçim
Gerekçeleri}\label{kalux131p-seuxe7im-gerekuxe7eleri}

\paragraph{Neden MVC?}\label{neden-mvc}

\begin{itemize}
\tightlist
\item
  \textbf{Separation of concerns:} Frontend geliştiricisi API'yi
  bilmeden UI değiştirebilir
\item
  \textbf{Testability:} Controller logic unit test edilebilir
\item
  \textbf{Reusability:} Aynı API farklı frontend'lerden kullanılabilir
  (web, mobile, CLI)
\end{itemize}

\paragraph{Neden Ensemble Learning?}\label{neden-ensemble-learning}

\begin{itemize}
\tightlist
\item
  \textbf{Single point of failure yok:} Bir model başarısız olsa
  diğerleri çalışır
\item
  \textbf{Accuracy boost:} Ensemble genellikle tek modelden daha iyi
  performans gösterir
\item
  \textbf{Explainability:} Hangi modelin nasıl karar verdiği görülebilir
  (XAI/LIME)
\end{itemize}

\paragraph{Neden Cache-Aside?}\label{neden-cache-aside}

\begin{itemize}
\tightlist
\item
  \textbf{Dashboard yükleme hızı:} \textasciitilde1s →
  \textasciitilde200ms improvement
\item
  \textbf{Database load reduction:} Sık sorgular cache'ten karşılanır
\item
  \textbf{Simplicity:} Daha karmaşık write-through pattern'lere gerek
  yok
\end{itemize}

\begin{center}\rule{0.5\linewidth}{0.5pt}\end{center}

\subsection{5. SİSTEM ÖZELLİKLERİ VE KULLANICI
ARAYÜZÜ}\label{sistem-uxf6zellikleri-ve-kullanici-arayuxfczuxfc}

\subsubsection{5.1 Ana Panel (Dashboard)}\label{ana-panel-dashboard}

\textbf{Amaç:} Sistemin genel durumunu ve tehdit istatistiklerini tek
bakışta görüntülemek.

\textbf{Dashboard Bileşenleri:}

{\def\LTcaptype{none} % do not increment counter
\begin{longtable}[]{@{}
  >{\raggedright\arraybackslash}p{(\linewidth - 4\tabcolsep) * \real{0.3913}}
  >{\raggedright\arraybackslash}p{(\linewidth - 4\tabcolsep) * \real{0.3043}}
  >{\raggedright\arraybackslash}p{(\linewidth - 4\tabcolsep) * \real{0.3043}}@{}}
\toprule\noalign{}
\begin{minipage}[b]{\linewidth}\raggedright
Bileşen
\end{minipage} & \begin{minipage}[b]{\linewidth}\raggedright
Konum
\end{minipage} & \begin{minipage}[b]{\linewidth}\raggedright
İşlev
\end{minipage} \\
\midrule\noalign{}
\endhead
\bottomrule\noalign{}
\endlastfoot
E-posta Analizi Kartı & Sol üst & Toplam analiz edilen e-posta ve
phishing oranı \\
Web Anomali Kartı & Orta üst & Web log analiz sayısı ve anomali oranı \\
Toplam Tehdit Kartı & Sağ üst & Tüm vektörlerden tespit edilen tehdit
sayısı \\
Sistem Durumu Kartı & Sağ üst & API ve model yükleme durumu (\%
olarak) \\
Tehdit Dağılımı Grafiği & Sol alt & Donut chart: Phishing vs Legitimate
dağılımı \\
Model Performans Grafiği & Sağ alt & Bar chart: Model bazlı doğruluk
oranları \\
\end{longtable}
}

\textbf{Üst Menü Butonları:} - \textbf{Generate Demo Data:} Test amaçlı
örnek veri seti oluşturur (30 e-posta + 30 web log + 5 koordineli
saldırı) - \textbf{Clear History:} Tüm geçmiş verileri siler ve
istatistikleri sıfırlar - \textbf{Tema Değiştir (☀/🌙):}
Aydınlık/Karanlık mod arasında geçiş yapar - \textbf{Dil Değiştir
(TR/EN):} Arayüz dilini değiştirir

\subsubsection{5.2 E-posta Analizi}\label{e-posta-analizi}

\textbf{Amaç:} E-posta içeriklerini üç farklı yapay zeka modeli ile
analiz ederek phishing tespiti yapmak.

\textbf{Giriş Alanları:} - \textbf{Email Subject (Konu):} E-postanın
konu satırı. Phishing e-postaları genellikle aciliyet içeren konular
kullanır. - \textbf{From Address (Gönderen):} Gönderen e-posta adresi.
Şüpheli domain'ler tespit edilir. - \textbf{Email Body (İçerik):}
E-postanın tam metin içeriği. Ana analiz bu alan üzerinde yapılır.

\textbf{Analiz Sonuç Bölümü:} Her üç model için ayrı ayrı sonuçlar
gösterilir: - \textbf{BERT Panel:} En yüksek doğruluklu model. Bağlamsal
anlam çıkarımı yapar. - \textbf{FastText Panel:} En hızlı model. Yüksek
hacimli işlemler için idealdir. - \textbf{TF-IDF Panel:} Baseline model.
Açıklanabilir sonuçlar sunar (LIME).

\textbf{Sonuç Gösterimi:} Her model için tahmin (PHISHING/LEGITIMATE),
güven skoru (0-100\%), risk seviyesi (Critical/High/Medium/Low) ve öne
çıkan özellikler gösterilir.

\subsubsection{5.3 Web Log Analizi}\label{web-log-analizi}

\textbf{Amaç:} Web sunucu loglarını analiz ederek SQL Injection, XSS ve
DDoS gibi saldırı girişimlerini tespit etmek.

\textbf{Giriş Alanları:} - \textbf{IP Address:} İstemci IP adresi.
Bilinen kötü niyetli IP'ler işaretlenir. - \textbf{HTTP Method:} GET,
POST, PUT, DELETE vb. Anomali tespitinde kullanılır. - \textbf{Request
Path:} İstenen URL yolu. SQL injection kalıpları aranır. -
\textbf{Status Code:} HTTP yanıt kodu. Çok sayıda 401/403 şüphelidir. -
\textbf{User Agent:} Tarayıcı/bot bilgisi. Otomatik araçlar tespit
edilir (sqlmap, nikto vb.). - \textbf{Response Size:} Yanıt boyutu.
Anormal boyutlar veri sızıntısına işaret edebilir.

\textbf{Analiz Algoritması:} - \textbf{Kullanılan Model:} Isolation
Forest algoritması (anomali tespiti için optimize edilmiştir) -
\textbf{Tespit Edilen Saldırı Türleri:} SQL Injection, Cross-Site
Scripting (XSS), Path Traversal, Brute Force, Bot/Crawler Activity, DDoS
Patterns

\subsubsection{5.4 Korelasyon Analizi}\label{korelasyon-analizi}

\textbf{Amaç:} E-posta ve web tehditlerini zaman ve IP bazında
ilişkilendirerek koordineli saldırıları tespit etmek.

\textbf{Korelasyon Metrikleri:} - \textbf{Korelasyon Skoru:} Pearson
korelasyon katsayısı (-1 ile +1 arası). Pozitif değerler eş zamanlı
artışı gösterir. - \textbf{Korelasyon Gücü:} Very Weak / Weak / Moderate
/ Strong olarak sınıflandırma. - \textbf{Koordineli Saldırı Sayısı:}
Aynı saat diliminde hem e-posta hem web tehdidi tespit edilen durumlar.
- \textbf{IP Boost:} Aynı IP'den hem phishing hem web saldırısı
geldiğinde eklenen bonus skor.

\textbf{Grafikler:} - \textbf{Threat Timeline Correlation:} Saat bazında
e-posta ve web tehditlerinin çakışma grafiği. - \textbf{Email vs Web
Comparison:} İki vektörün karşılaştırmalı bar chart'ı. -
\textbf{Correlation Heatmap:} Tehdit korelasyonunun ısı haritası
görselleştirmesi.

\subsubsection{5.5 Model
Karşılaştırma}\label{model-karux15fux131laux15ftux131rma}

\textbf{Amaç:} Tüm yapay zeka modellerinin performans metriklerini
karşılaştırmalı olarak görüntülemek.

{\def\LTcaptype{none} % do not increment counter
\begin{longtable}[]{@{}lllll@{}}
\toprule\noalign{}
Model & Accuracy & Precision & Recall & F1-Score \\
\midrule\noalign{}
\endhead
\bottomrule\noalign{}
\endlastfoot
BERT (DistilBERT) & \%94-97 & \%95 & \%93 & \%94 \\
FastText & \%90-94 & \%92 & \%90 & \%91 \\
TF-IDF + Random Forest & \%89.75 & \%90 & \%88 & \%89 \\
Isolation Forest (Web) & \%92+ & N/A & N/A & N/A \\
\end{longtable}
}

\subsubsection{5.6 Raporlar ve Ayarlar}\label{raporlar-ve-ayarlar}

\textbf{Raporlar - Dışa Aktarma:} - \textbf{Export to Excel:} Tüm tehdit
verilerini .xlsx formatında indirir. Pivot tablo oluşturmaya uygun. -
\textbf{Export to JSON:} API entegrasyonu için JSON formatında dışa
aktarır.

\textbf{Raporlar - İçe Aktarma:} - \textbf{Import from Excel:} Toplu
e-posta veya web log verisi yüklemek için. - \textbf{Import from JSON:}
Programatik veri aktarımı için.

\textbf{Ayar Seçenekleri:}

{\def\LTcaptype{none} % do not increment counter
\begin{longtable}[]{@{}
  >{\raggedright\arraybackslash}p{(\linewidth - 4\tabcolsep) * \real{0.2857}}
  >{\raggedright\arraybackslash}p{(\linewidth - 4\tabcolsep) * \real{0.2381}}
  >{\raggedright\arraybackslash}p{(\linewidth - 4\tabcolsep) * \real{0.4762}}@{}}
\toprule\noalign{}
\begin{minipage}[b]{\linewidth}\raggedright
Ayar
\end{minipage} & \begin{minipage}[b]{\linewidth}\raggedright
Tür
\end{minipage} & \begin{minipage}[b]{\linewidth}\raggedright
Açıklama
\end{minipage} \\
\midrule\noalign{}
\endhead
\bottomrule\noalign{}
\endlastfoot
Dark Mode & Toggle & Karanlık/Aydınlık tema tercihi. Tarayıcı kapatılsa
da korunur. \\
Language & Checkbox & Arayüz dili: İngilizce (varsayılan) veya
Türkçe. \\
Detection Threshold & Slider & Phishing tespit eşiği (0.0 - 1.0). Düşük
değer = daha hassas. \\
High Risk Alerts & Toggle & Yüksek riskli tehditler için anlık
bildirim. \\
\end{longtable}
}

\textbf{Ayar Kalıcılığı:} Tüm ayarlar hem localStorage (anlık tepki) hem
de PostgreSQL veritabanına (kalıcı depolama) kaydedilir.

\begin{center}\rule{0.5\linewidth}{0.5pt}\end{center}

\subsection{6. YAPAY ZEKA MODELLERİ}\label{yapay-zeka-modelleri}

\subsubsection{6.1 BERT (DistilBERT)}\label{bert-distilbert}

{\def\LTcaptype{none} % do not increment counter
\begin{longtable}[]{@{}
  >{\raggedright\arraybackslash}p{(\linewidth - 2\tabcolsep) * \real{0.5625}}
  >{\raggedright\arraybackslash}p{(\linewidth - 2\tabcolsep) * \real{0.4375}}@{}}
\toprule\noalign{}
\begin{minipage}[b]{\linewidth}\raggedright
Özellik
\end{minipage} & \begin{minipage}[b]{\linewidth}\raggedright
Değer
\end{minipage} \\
\midrule\noalign{}
\endhead
\bottomrule\noalign{}
\endlastfoot
Mimari & Transformer tabanlı, bidirectional encoder \\
Kaynak & Hugging Face Transformers kütüphanesi \\
Eğitim Verisi & 31,000+ e-posta (CEAS, Enron, Nigerian Fraud,
SpamAssassin) \\
Doğruluk & \%94-97 \\
İşlem Süresi & \textasciitilde45ms / e-posta \\
Avantajı & Bağlamsal anlam çıkarımı, kelime ilişkilerini anlama \\
Dezavantajı & Diğer modellere göre daha yavaş \\
\end{longtable}
}

\subsubsection{6.2 FastText}\label{fasttext}

{\def\LTcaptype{none} % do not increment counter
\begin{longtable}[]{@{}ll@{}}
\toprule\noalign{}
Özellik & Değer \\
\midrule\noalign{}
\endhead
\bottomrule\noalign{}
\endlastfoot
Mimari & Word embedding + Linear classifier \\
Kaynak & Facebook Research \\
Model Boyutu & 881 MB \\
Doğruluk & \%90-94 \\
İşlem Süresi & \textless1ms / e-posta \\
Avantajı & Çok hızlı, büyük hacimler için ideal \\
Dezavantajı & Karmaşık bağlam anlamada BERT'e göre zayıf \\
\end{longtable}
}

\subsubsection{6.3 TF-IDF + Random Forest}\label{tf-idf-random-forest}

{\def\LTcaptype{none} % do not increment counter
\begin{longtable}[]{@{}ll@{}}
\toprule\noalign{}
Özellik & Değer \\
\midrule\noalign{}
\endhead
\bottomrule\noalign{}
\endlastfoot
Mimari & TF-IDF vektörizasyon + Random Forest ensemble \\
Eğitim & SMOTE ile dengelenmiş veri seti \\
Doğruluk & \%89.75 \\
ROC-AUC & \%97.50 \\
İşlem Süresi & \textasciitilde25ms / e-posta \\
Avantajı & Açıklanabilir sonuçlar, özellik önem sıralaması (LIME) \\
Dezavantajı & Deep learning modellere göre düşük doğruluk \\
\end{longtable}
}

\subsubsection{6.4 Isolation Forest (Web
Analizi)}\label{isolation-forest-web-analizi}

{\def\LTcaptype{none} % do not increment counter
\begin{longtable}[]{@{}ll@{}}
\toprule\noalign{}
Özellik & Değer \\
\midrule\noalign{}
\endhead
\bottomrule\noalign{}
\endlastfoot
Mimari & Unsupervised anomaly detection \\
Kullanım & Web log anomali tespiti \\
Doğruluk & \%92+ \\
Tespit Edilen Saldırılar & SQL Injection, XSS, DDoS, Brute Force \\
\end{longtable}
}

\begin{center}\rule{0.5\linewidth}{0.5pt}\end{center}

\subsection{7. TEST METODOLOJİSİ VE
SONUÇLARI}\label{test-metodolojisi-ve-sonuuxe7lari}

\subsubsection{7.1 Test Stratejisi ve
Amacı}\label{test-stratejisi-ve-amacux131}

CyberGuard için tasarlanan test stratejisi, sistemin temel güvenlik
fonksiyonlarının doğruluğunu ve kullanıcı deneyimini öncelikli olarak
hedeflemiştir.

\textbf{Test Odak Alanları:}

{\def\LTcaptype{none} % do not increment counter
\begin{longtable}[]{@{}
  >{\raggedright\arraybackslash}p{(\linewidth - 4\tabcolsep) * \real{0.4231}}
  >{\raggedright\arraybackslash}p{(\linewidth - 4\tabcolsep) * \real{0.2308}}
  >{\raggedright\arraybackslash}p{(\linewidth - 4\tabcolsep) * \real{0.3462}}@{}}
\toprule\noalign{}
\begin{minipage}[b]{\linewidth}\raggedright
Test Tipi
\end{minipage} & \begin{minipage}[b]{\linewidth}\raggedright
Amaç
\end{minipage} & \begin{minipage}[b]{\linewidth}\raggedright
Öncelik
\end{minipage} \\
\midrule\noalign{}
\endhead
\bottomrule\noalign{}
\endlastfoot
Accuracy Testi & ML modellerinin phishing/legitimate ayrımını doğru
yapması & 🔴 Kritik \\
Functional Testi & Tüm UI bileşenlerinin ve API endpoint'lerinin
çalışması & 🔴 Kritik \\
Integration Testi & Backend-Database-Cache entegrasyonu & 🟡 Yüksek \\
Usability Testi & Tema, dil, ayar kalıcılığı & 🟢 Orta \\
\end{longtable}
}

\subsubsection{7.2 Neden Bu Testler
Seçildi?}\label{neden-bu-testler-seuxe7ildi}

\paragraph{Neden Accuracy
Ölçüldü?}\label{neden-accuracy-uxf6luxe7uxfclduxfc}

ML-based siber güvenlik sistemlerinde \textbf{False Positive} ve
\textbf{False Negative} oranları kritik öneme sahiptir: - \textbf{False
Negative (kaçırılan phishing):} Güvenlik açığı, potansiyel data breach -
\textbf{False Positive (yanlış alarm):} Operasyonel verimlilik kaybı,
user trust azalması

Bu nedenle accuracy, precision, recall ve F1-score metrikleri detaylı
olarak ölçülmüştür.

\paragraph{Neden Latency Detaylı
Ölçülmedi?}\label{neden-latency-detaylux131-uxf6luxe7uxfclmedi}

\begin{itemize}
\tightlist
\item
  \textbf{Kullanım Senaryosu:} CyberGuard, real-time stream processing
  değil, \textbf{on-demand analiz} sistemidir
\item
  \textbf{Acceptable Threshold:} 1-2 saniye response time, kullanıcı
  deneyimi için kabul edilebilir
\item
  \textbf{Gelecek Çalışma:} Production deployment'ta P95/P99 latency
  Grafana ile monitör edilmeli
\end{itemize}

\paragraph{Neden Load Test
Yapılmadı?}\label{neden-load-test-yapux131lmadux131}

\begin{itemize}
\tightlist
\item
  \textbf{Hedef Kitle:} Orta ölçekli kurumlar (10-100 concurrent user)
\item
  \textbf{Current Capacity:} Flask + Gunicorn (4 worker) bu senaryoyu
  karşılamaktadır
\item
  \textbf{Gelecek Çalışma:} Kurumsal deployment öncesi Apache JMeter ile
  load test yapılmalı
\end{itemize}

\subsubsection{7.3 Fonksiyonel Test
Sonuçları}\label{fonksiyonel-test-sonuuxe7larux131}

{\def\LTcaptype{none} % do not increment counter
\begin{longtable}[]{@{}ll@{}}
\toprule\noalign{}
Test & Sonuç \\
\midrule\noalign{}
\endhead
\bottomrule\noalign{}
\endlastfoot
Dashboard yükleme ve grafikler & ✅ BAŞARILI \\
E-posta phishing tespiti (3 model) & ✅ BAŞARILI \\
E-posta legitimate sınıflandırma & ✅ BAŞARILI \\
Web log anomali tespiti & ✅ BAŞARILI \\
Web log normal trafik sınıflandırma & ✅ BAŞARILI \\
Korelasyon analizi hesaplama & ✅ BAŞARILI \\
Koordineli saldırı tespiti & ✅ BAŞARILI \\
Tema değiştirme ve kalıcılık & ✅ BAŞARILI \\
Dil değiştirme (TR/EN) & ✅ BAŞARILI \\
Ayar kaydetme ve yükleme & ✅ BAŞARILI \\
\end{longtable}
}

\subsubsection{7.4 Performans Metrikleri}\label{performans-metrikleri}

{\def\LTcaptype{none} % do not increment counter
\begin{longtable}[]{@{}ll@{}}
\toprule\noalign{}
Metrik & Ölçüm \\
\midrule\noalign{}
\endhead
\bottomrule\noalign{}
\endlastfoot
API ortalama yanıt süresi & \textasciitilde200ms \\
BERT analiz süresi & \textasciitilde45ms \\
FastText analiz süresi & \textless1ms \\
TF-IDF analiz süresi & \textasciitilde25ms \\
Dashboard tam yükleme & \textless1 saniye \\
Demo data oluşturma (60 kayıt) & \textasciitilde2 saniye \\
\end{longtable}
}

\begin{center}\rule{0.5\linewidth}{0.5pt}\end{center}

\subsection{8. MODEL KARŞILAŞTIRMASI VE TRADE-OFF
ANALİZİ}\label{model-karux15filaux15ftirmasi-ve-trade-off-analizi}

\subsubsection{8.1 Performans
Karşılaştırması}\label{performans-karux15fux131laux15ftux131rmasux131}

{\def\LTcaptype{none} % do not increment counter
\begin{longtable}[]{@{}llllll@{}}
\toprule\noalign{}
Model & Accuracy & Precision & Recall & F1-Score & Inference Time \\
\midrule\noalign{}
\endhead
\bottomrule\noalign{}
\endlastfoot
BERT (DistilBERT) & \%94-97 & \%95 & \%93 & \%94 &
\textasciitilde45ms \\
FastText & \%90-94 & \%92 & \%90 & \%91 & \textless1ms \\
TF-IDF + Random Forest & \%89.75 & \%90 & \%88 & \%89 &
\textasciitilde25ms \\
\end{longtable}
}

\subsubsection{8.2 BERT Neden Daha İyi Performans
Gösterdi?}\label{bert-neden-daha-iyi-performans-guxf6sterdi}

\begin{enumerate}
\def\labelenumi{\arabic{enumi}.}
\item
  \textbf{Contextual Understanding:} BERT, kelimelerin bağlamını anlar.
  `Bank' kelimesi `river bank' ve `bank account' için farklı embedding
  üretir.
\item
  \textbf{Transfer Learning:} 1.5 milyar kelime üzerinde pre-train
  edilmiş model, phishing dataset'inde fine-tune edilmiştir.
\item
  \textbf{Subword Tokenization:} `PayPaI' (I harfi ile sahte PayPal)
  gibi typosquatting saldırılarını yakalayabilir.
\item
  \textbf{Attention Mechanism:} Hangi kelimelerin phishing tespitinde
  önemli olduğunu öğrenir (`urgent', `verify', `click').
\end{enumerate}

\subsubsection{8.3 Hız vs Doğruluk
Trade-off}\label{hux131z-vs-doux11fruluk-trade-off}

\begin{verbatim}
    HIZLI ◄────────────────────────────────► YAVAŞ
       │                                       │
    FastText                                 BERT
     (<1ms)                                 (45ms)
       │                                       │
       ▼                                       ▼
    %90-94 Acc                            %94-97 Acc
               ┌─────────────┐
               │   TF-IDF    │
               │   (25ms)    │
               │ %89.75 Acc  │
               └─────────────┘
\end{verbatim}

\textbf{Kullanım Senaryosu Önerileri:}

{\def\LTcaptype{none} % do not increment counter
\begin{longtable}[]{@{}
  >{\raggedright\arraybackslash}p{(\linewidth - 4\tabcolsep) * \real{0.2647}}
  >{\raggedright\arraybackslash}p{(\linewidth - 4\tabcolsep) * \real{0.4706}}
  >{\raggedright\arraybackslash}p{(\linewidth - 4\tabcolsep) * \real{0.2647}}@{}}
\toprule\noalign{}
\begin{minipage}[b]{\linewidth}\raggedright
Senaryo
\end{minipage} & \begin{minipage}[b]{\linewidth}\raggedright
Önerilen Model
\end{minipage} & \begin{minipage}[b]{\linewidth}\raggedright
Gerekçe
\end{minipage} \\
\midrule\noalign{}
\endhead
\bottomrule\noalign{}
\endlastfoot
Real-time Email Gateway & FastText & Yüksek throughput gerekli,
\textless1ms latency \\
Kritik Güvenlik Analizi & BERT & Accuracy kritik, latency kabul
edilebilir \\
Balanced / Genel Kullanım & TF-IDF + RF & İyi denge, açıklanabilirlik
(LIME) \\
Ensemble (Production) & Üçü birlikte & En yüksek accuracy, weighted
voting \\
\end{longtable}
}

\subsubsection{8.4 False Positive / False Negative
Analizi}\label{false-positive-false-negative-analizi}

\paragraph{False Positive Senaryoları (Meşru → Phishing Olarak Yanlış
Tespit)}\label{false-positive-senaryolarux131-meux15fru-phishing-olarak-yanlux131ux15f-tespit}

{\def\LTcaptype{none} % do not increment counter
\begin{longtable}[]{@{}
  >{\raggedright\arraybackslash}p{(\linewidth - 2\tabcolsep) * \real{0.4737}}
  >{\raggedright\arraybackslash}p{(\linewidth - 2\tabcolsep) * \real{0.5263}}@{}}
\toprule\noalign{}
\begin{minipage}[b]{\linewidth}\raggedright
Senaryo
\end{minipage} & \begin{minipage}[b]{\linewidth}\raggedright
Açıklama
\end{minipage} \\
\midrule\noalign{}
\endhead
\bottomrule\noalign{}
\endlastfoot
Agresif Marketing E-postaları & `Limited time offer!', `Act now!' gibi
ifadeler \\
IT Departmanı Uyarıları & `Your password will expire' gibi legitimate
sistem mesajları \\
Kısa Mesajlar & `Hey, how are you?' gibi çok kısa mesajlarda model
güvensiz olabiliyordu (v2.0'da düzeltildi) \\
\end{longtable}
}

\textbf{Mitigation Stratejileri:} - Whitelist domain desteği - Threshold
ayarı - Human-in-the-loop review süreci

\paragraph{False Negative Senaryoları (Phishing → Meşru Olarak
Kaçırılan)}\label{false-negative-senaryolarux131-phishing-meux15fru-olarak-kauxe7ux131rux131lan}

{\def\LTcaptype{none} % do not increment counter
\begin{longtable}[]{@{}
  >{\raggedright\arraybackslash}p{(\linewidth - 2\tabcolsep) * \real{0.4737}}
  >{\raggedright\arraybackslash}p{(\linewidth - 2\tabcolsep) * \real{0.5263}}@{}}
\toprule\noalign{}
\begin{minipage}[b]{\linewidth}\raggedright
Senaryo
\end{minipage} & \begin{minipage}[b]{\linewidth}\raggedright
Açıklama
\end{minipage} \\
\midrule\noalign{}
\endhead
\bottomrule\noalign{}
\endlastfoot
Hedefli Spear Phishing & Kişiselleştirilmiş, phishing keyword içermeyen
saldırılar \\
Zero-Day Phishing & Yeni kampanyalar, training data'da olmayan
pattern'ler \\
Homograph Saldırıları & `pаypal.com' (Kiril `а' karakteri) gibi punycode
saldırıları \\
\end{longtable}
}

\textbf{Mitigation Stratejileri:} - VirusTotal API ile URL reputation
check - Domain age check - Sürekli model retraining

\subsubsection{8.5 Concept Drift Riski ve Çözüm
Stratejileri}\label{concept-drift-riski-ve-uxe7uxf6zuxfcm-stratejileri}

\textbf{Concept Drift Nedir?} Phishing saldırıları sürekli evrilir.
2025'te etkili olan phishing pattern'leri 2026'da değişmiş olabilir.

\textbf{Risk Faktörleri:} - Yeni phishing kampanya temaları
(AI-generated phishing, deepfake) - Yeni sosyal mühendislik teknikleri -
Değişen e-posta formatları

\textbf{Önerilen Stratejiler:}

{\def\LTcaptype{none} % do not increment counter
\begin{longtable}[]{@{}
  >{\raggedright\arraybackslash}p{(\linewidth - 4\tabcolsep) * \real{0.3448}}
  >{\raggedright\arraybackslash}p{(\linewidth - 4\tabcolsep) * \real{0.3448}}
  >{\raggedright\arraybackslash}p{(\linewidth - 4\tabcolsep) * \real{0.3103}}@{}}
\toprule\noalign{}
\begin{minipage}[b]{\linewidth}\raggedright
Strateji
\end{minipage} & \begin{minipage}[b]{\linewidth}\raggedright
Açıklama
\end{minipage} & \begin{minipage}[b]{\linewidth}\raggedright
Periyot
\end{minipage} \\
\midrule\noalign{}
\endhead
\bottomrule\noalign{}
\endlastfoot
Periyodik Retraining & Model güncellemesi & Her 3-6 ayda bir \\
Active Learning & False positive/negative feedback'lerden öğrenme &
Sürekli \\
Ensemble Diversification & Farklı feature'lara dayanan modeller kullanma
& Initial design \\
Continuous Monitoring & Accuracy metrikleri düşüşü için alerting &
Günlük \\
\end{longtable}
}

\begin{center}\rule{0.5\linewidth}{0.5pt}\end{center}

\subsection{9. API REFERANSI}\label{api-referansi}

{\def\LTcaptype{none} % do not increment counter
\begin{longtable}[]{@{}
  >{\raggedright\arraybackslash}p{(\linewidth - 4\tabcolsep) * \real{0.3571}}
  >{\raggedright\arraybackslash}p{(\linewidth - 4\tabcolsep) * \real{0.2857}}
  >{\raggedright\arraybackslash}p{(\linewidth - 4\tabcolsep) * \real{0.3571}}@{}}
\toprule\noalign{}
\begin{minipage}[b]{\linewidth}\raggedright
Endpoint
\end{minipage} & \begin{minipage}[b]{\linewidth}\raggedright
Method
\end{minipage} & \begin{minipage}[b]{\linewidth}\raggedright
Açıklama
\end{minipage} \\
\midrule\noalign{}
\endhead
\bottomrule\noalign{}
\endlastfoot
\texttt{/api/health} & GET & Sistem sağlık kontrolü \\
\texttt{/api/models/status} & GET & Model yükleme durumları \\
\texttt{/api/email/analyze} & POST & TF-IDF ile e-posta analizi \\
\texttt{/api/email/analyze/bert} & POST & BERT ile e-posta analizi \\
\texttt{/api/email/analyze/fasttext} & POST & FastText ile e-posta
analizi \\
\texttt{/api/email/analyze/hybrid} & POST & Tüm modeller ile analiz
(Ensemble) \\
\texttt{/api/predict/web} & POST & Web log anomali analizi \\
\texttt{/api/correlation/analyze} & GET & Korelasyon analizi \\
\texttt{/api/dashboard/stats} & GET & Dashboard istatistikleri \\
\texttt{/api/reports/export/excel} & GET & Excel dışa aktarma \\
\texttt{/api/reports/export/json} & GET & JSON dışa aktarma \\
\texttt{/api/settings} & GET/POST & Ayarları getir/kaydet \\
\texttt{/api/demo/generate} & POST & Demo veri oluştur \\
\texttt{/api/database/clear} & POST & Verileri temizle \\
\texttt{/api/virustotal/check-ip/\textless{}ip\textgreater{}} & GET &
VirusTotal IP kontrolü \\
\texttt{/api/virustotal/check-domain/\textless{}domain\textgreater{}} &
GET & VirusTotal domain kontrolü \\
\end{longtable}
}

\begin{center}\rule{0.5\linewidth}{0.5pt}\end{center}

\subsection{10. KURULUM VE YAPILANDIRMA}\label{kurulum-ve-yapilandirma}

\subsubsection{10.1 Sistem Gereksinimleri}\label{sistem-gereksinimleri}

{\def\LTcaptype{none} % do not increment counter
\begin{longtable}[]{@{}ll@{}}
\toprule\noalign{}
Bileşen & Gereksinim \\
\midrule\noalign{}
\endhead
\bottomrule\noalign{}
\endlastfoot
İşletim Sistemi & Windows 10+, Linux, macOS \\
Python & 3.8 veya üzeri \\
RAM & Minimum 4GB, önerilen 8GB \\
Disk & 2GB (uygulama + modeller) \\
Docker & 24.0+ (konteyner dağıtımı için) \\
\end{longtable}
}

\subsubsection{10.2 Docker ile Kurulum}\label{docker-ile-kurulum}

\begin{Shaded}
\begin{Highlighting}[]
\CommentTok{\# 1. Projeyi klonlayın}
\FunctionTok{git}\NormalTok{ clone https://github.com/TheLastKhan/UnifiedCyberThreatDetectionSystem.git}
\BuiltInTok{cd}\NormalTok{ UnifiedCyberThreatDetectionSystem}

\CommentTok{\# 2. Docker container\textquotesingle{}ları başlatın}
\ExtensionTok{docker{-}compose}\NormalTok{ up }\AttributeTok{{-}d}

\CommentTok{\# 3. Durumu kontrol edin}
\ExtensionTok{docker{-}compose}\NormalTok{ ps}

\CommentTok{\# 4. Servislere erişin}
\CommentTok{\# Dashboard: http://localhost:5000}
\CommentTok{\# Grafana: http://localhost:3000 (admin/admin)}
\CommentTok{\# Prometheus: http://localhost:9090}
\end{Highlighting}
\end{Shaded}

\subsubsection{10.3 Docker Container
Yapısı}\label{docker-container-yapux131sux131}

{\def\LTcaptype{none} % do not increment counter
\begin{longtable}[]{@{}llll@{}}
\toprule\noalign{}
Container & Port & İşlev & Bağımlılık \\
\midrule\noalign{}
\endhead
\bottomrule\noalign{}
\endlastfoot
threat-detection-api & 5000 & Flask API + ML Modelleri & db, cache \\
threat-detection-db & 5432 & PostgreSQL Veritabanı & - \\
threat-detection-cache & 6379 & Redis Önbellek & - \\
threat-detection-nginx & 80, 443 & Reverse Proxy, SSL & api \\
threat-detection-prometheus & 9090 & Metrik Toplama & api \\
threat-detection-grafana & 3000 & Görselleştirme Paneli & prometheus \\
\end{longtable}
}

\subsubsection{10.4 Manuel Kurulum}\label{manuel-kurulum}

\begin{Shaded}
\begin{Highlighting}[]
\CommentTok{\# 1. Virtual environment oluşturun}
\ExtensionTok{python} \AttributeTok{{-}m}\NormalTok{ venv venv}
\BuiltInTok{source}\NormalTok{ venv/bin/activate  }\CommentTok{\# Windows: venv\textbackslash{}Scripts\textbackslash{}activate}

\CommentTok{\# 2. Bağımlılıkları yükleyin}
\ExtensionTok{pip}\NormalTok{ install }\AttributeTok{{-}r}\NormalTok{ requirements.txt}

\CommentTok{\# 3. Dashboard\textquotesingle{}u başlatın}
\ExtensionTok{python}\NormalTok{ run\_dashboard.py}

\CommentTok{\# 4. Tarayıcıda açın: http://localhost:5000}
\end{Highlighting}
\end{Shaded}

\begin{center}\rule{0.5\linewidth}{0.5pt}\end{center}

\subsection{11. SONUÇ}\label{sonuuxe7}

CyberGuard, modern yapay zeka teknolojilerini kullanarak kapsamlı bir
siber güvenlik çözümü sunmaktadır.

\subsubsection{Sistemin Temel
Başarıları}\label{sistemin-temel-baux15farux131larux131}

✅ \textbf{3 farklı ML modeli} ile yüksek doğrulukta phishing tespiti
(ensemble: BERT 0.5, FastText 0.3, TF-IDF 0.2)

✅ \textbf{Modüler, servis-odaklı mimari} ile bakım kolaylığı ve
bağımsız model geliştirme imkanı

✅ \textbf{Bilinen tasarım kalıpları} (MVC, Event-Driven, Ensemble,
Cache-Aside, Factory, Strategy, Façade, Circuit Breaker) ile sağlam
altyapı

✅ \textbf{Gerçek zamanlı korelasyon analizi} ile koordineli saldırı
tespiti (e-posta + web vektörleri)

✅ \textbf{Trade-off bilinci} ile kullanım senaryosuna uygun model
seçimi önerileri

✅ \textbf{Docker ile kolay dağıtım} ve production-ready altyapı (6
container: API, DB, Cache, Nginx, Prometheus, Grafana)

✅ \textbf{Açıklanabilir AI (XAI)} - LIME ile model kararlarının
görselleştirilmesi

✅ \textbf{False positive/negative analizi} ve mitigation stratejileri

✅ \textbf{Concept drift riski} farkındalığı ve periyodik retraining
planı

\subsubsection{Hedef Kitle ve
Ölçeklenebilirlik}\label{hedef-kitle-ve-uxf6luxe7eklenebilirlik}

Sistem, özellikle \textbf{orta ölçekli kurumlar} (10-100 concurrent
user) için optimize edilmiştir. Gerektiğinde: - Horizontal scaling için
model inference ayrı container'lara taşınabilir - TensorFlow Serving
veya TorchServe ile dedicated inference server kurulabilir - Kubernetes
ile container orchestration yapılabilir

\begin{center}\rule{0.5\linewidth}{0.5pt}\end{center}

\textbf{© 2025-2026 CyberGuard Project Team}

\begin{center}\rule{0.5\linewidth}{0.5pt}\end{center}

\emph{Bu rapor, CyberGuard projesinin teknik dokümantasyonu ve akademik
değerlendirme amaçlı hazırlanmıştır.}

\end{document}
