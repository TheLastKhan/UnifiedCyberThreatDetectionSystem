% ====== REALISTIC CONSTRAINTS AND CONDITIONS ======
\section{Realistic Constraints and Conditions}

\subsection{Sustainable Development Goal}

\textbf{Selected SDG:} Goal 9 -- Industry, Innovation and Infrastructure

\textbf{Target 9.c:} Significantly increase access to information and communications technology and strive to provide universal and affordable access to the Internet.

This project directly contributes to SDG Goal 9 by enhancing the security and reliability of digital infrastructure, which is fundamental to providing safe and universal access to information and communication technologies. As cyber threats continue to evolve and multiply, they pose significant barriers to the adoption and trust of digital technologies, particularly in developing regions where cybersecurity expertise and resources are limited.

\textbf{Direct Contributions:}

\textbf{Infrastructure Security:} By providing an intelligent, explainable, and cost-effective cybersecurity solution, this platform helps organizations of all sizes protect their digital infrastructure. Small and medium enterprises (SMEs) and organizations in developing countries often lack the resources to implement comprehensive security solutions. Our platform's design emphasizes efficiency and interpretability, making advanced threat detection accessible to organizations with limited security expertise.

\textbf{Knowledge Transfer and Capacity Building:} The explainable AI component of this platform serves an educational purpose beyond immediate threat detection. By providing clear, understandable explanations of why certain activities are flagged as malicious, the system helps security practitioners learn and improve their threat detection capabilities over time. This knowledge transfer is particularly valuable in regions where formal cybersecurity training is limited.

\textbf{Innovation in Technology:} This project demonstrates innovation in artificial intelligence application to cybersecurity, combining multiple detection mechanisms with intelligent correlation and explainable decision-making. This innovation contributes to the broader goal of advancing technological capabilities in the digital infrastructure domain.

\textbf{Resilience and Trust:} Secure digital infrastructure is essential for building trust in Internet-based services, including e-commerce, digital education, telehealth, and e-government. By improving threat detection and response capabilities, this platform contributes to building more resilient and trustworthy digital ecosystems, encouraging wider adoption of information and communication technologies.

\textbf{Sustainable Technology:} The platform is designed to be resource-efficient, utilizing machine learning algorithms that can operate effectively without requiring excessive computational resources. This consideration is important for sustainability and accessibility in regions with limited technological infrastructure.

\subsection{Effects on Health, Environment and the Problems of the Age}

\subsubsection{Health Impacts}

The digital health sector has experienced explosive growth, accelerated by the COVID-19 pandemic, with telemedicine, electronic health records, and health information exchanges becoming critical infrastructure. Cyber attacks on healthcare systems can have direct life-threatening consequences, as demonstrated by ransomware attacks that have shut down hospital operations and compromised patient data.

This platform contributes to health protection in several ways:

\textbf{Healthcare Infrastructure Protection:} By detecting and preventing cyber attacks on healthcare systems, the platform helps ensure continuity of critical medical services. The ability to identify coordinated attacks is particularly crucial in healthcare, where attackers often combine social engineering (phishing healthcare workers) with technical exploitation of medical systems.

\textbf{Patient Data Privacy:} Healthcare organizations are primary targets for data breaches due to the value of medical records on black markets. The platform's proactive threat detection helps protect sensitive patient information, supporting the right to health privacy mandated by regulations such as HIPAA and GDPR.

\textbf{Medical Device Security:} Modern medical devices are increasingly connected to hospital networks and the Internet, creating potential attack vectors. While this project focuses on email and web threats, its architecture can be extended to monitor communications involving medical IoT devices.

\subsubsection{Environmental Impacts}

The environmental impact of cybersecurity systems is an emerging concern as the energy consumption of data centers and computing infrastructure continues to grow.

\begin{itemize}
    \item \textbf{Resource Efficiency:} This platform has been designed with computational efficiency in mind. The Random Forest and Isolation Forest algorithms were chosen partly because they provide acceptable performance without requiring the massive computational resources of deep learning models. This efficiency translates to reduced energy consumption and lower carbon footprint.
    \item \textbf{Prevention of Environmental Cyber Attacks:} Critical infrastructure including power grids, water treatment facilities, and environmental monitoring systems are increasingly targeted by cyber attacks. The platform's correlation capabilities can help detect sophisticated attacks on environmental infrastructure before they cause physical damage.
    \item \textbf{Reduction of Hardware Waste:} By providing more accurate threat detection with fewer false positives, the platform reduces the need for excessive security appliances and redundant systems, contributing to reduced electronic waste.
\end{itemize}

\subsubsection{Problems of the Age in Engineering}

This project addresses several contemporary challenges in computer engineering and society:

\begin{itemize}
    \item \textbf{AI Ethics and Transparency:} The ``black box'' problem of AI systems represents one of the most significant ethical challenges in modern technology. This project directly addresses this concern by integrating explainability as a core feature, not an afterthought.
    \item \textbf{Digital Trust Crisis:} Society faces a growing crisis of trust in digital systems, fueled by frequent data breaches, privacy violations, and cyberattacks. This project contributes to rebuilding digital trust by providing more effective, transparent security solutions.
    \item \textbf{Skills Gap in Cybersecurity:} The global cybersecurity workforce shortage is estimated at over 3 million professionals. By making advanced threat detection more accessible and providing educational value through explainable decisions, this platform helps address the skills gap.
    \item \textbf{Complexity Management:} Modern IT environments have become extraordinarily complex, with organizations managing thousands of security alerts daily. This project addresses complexity through intelligent correlation and unified risk assessment.
\end{itemize}

\subsection{Legal Consequences}

The development and deployment of this cybersecurity platform intersects with multiple legal frameworks and regulatory requirements:

\subsubsection{Data Protection and Privacy Regulations}

\textbf{GDPR (General Data Protection Regulation):} The platform processes email content and web access logs, which may contain personal data. The system has been designed with privacy-by-design principles, ensuring that:
\begin{itemize}
    \item Personal data processing is minimized and limited to what is necessary for threat detection
    \item Data retention policies can be configured to comply with GDPR requirements
    \item Individuals' rights to access, rectify, and delete their data can be accommodated
    \item The explainability features support the GDPR's requirement for meaningful information about automated decision-making
\end{itemize}

\textbf{KVKK (Turkey Personal Data Protection Law):} As this project is developed in Turkey, compliance with KVKK is essential. The platform adheres to KVKK principles including lawfulness, fairness, transparency, purpose limitation, data minimization, accuracy, storage limitation, and security.

\subsubsection{Cybersecurity Regulations}

\textbf{NIS Directive (Network and Information Security):} For organizations operating in the EU, the platform supports compliance with NIS Directive requirements by providing incident detection and reporting capabilities.

\textbf{Sector-Specific Regulations:} Organizations in regulated sectors (healthcare, finance, energy) face additional cybersecurity requirements. The platform's comprehensive logging and reporting features support compliance with sector-specific standards such as:
\begin{itemize}
    \item HIPAA (Healthcare)
    \item PCI-DSS (Payment Card Industry)
    \item SOX (Financial Reporting)
    \item NERC-CIP (Energy Infrastructure)
\end{itemize}

\subsubsection{Liability and Legal Responsibility}

\textbf{Accuracy and False Positives:} While the platform achieves high accuracy rates, no AI system is perfect. Legal considerations include:
\begin{itemize}
    \item Clear documentation of system limitations and error rates
    \item Human oversight requirements for critical decisions
    \item Liability limitations in terms of service agreements
\end{itemize}

\textbf{Automated Decision-Making:} The platform makes automated decisions about threat classification. Legal frameworks increasingly require that such decisions be explainable and contestable, which the platform's XAI features directly address.

\subsubsection{Intellectual Property}

\textbf{Open Source Components:} The platform utilizes several open-source libraries (scikit-learn, LIME, Flask). All licensing requirements have been carefully reviewed and respected, with proper attribution provided.

\textbf{Patent Considerations:} While the individual components (ML algorithms, XAI techniques) are not novel inventions, the specific integration and correlation approach may have patentable aspects that could be explored in the future.

\newpage
