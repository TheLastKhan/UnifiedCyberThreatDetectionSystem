% ====== INTERDISCIPLINARY ASPECTS ======
\section{Interdisciplinary Aspects}

This project demonstrates significant interdisciplinary integration, combining expertise from multiple engineering and scientific domains:

\subsection{Computer Science and Software Engineering}

The core foundation of the project lies in computer science and software engineering disciplines:

\begin{itemize}
    \item \textbf{Algorithm Design:} Implementation of Random Forest and Isolation Forest algorithms for threat detection
    \item \textbf{Software Architecture:} Modular, layered design following SOLID principles
    \item \textbf{Web Development:} Flask-based RESTful API and interactive dashboard
    \item \textbf{Database Systems:} PostgreSQL for persistent storage, Redis for caching
    \item \textbf{DevOps:} Docker containerization, continuous integration
\end{itemize}

\subsection{Artificial Intelligence and Machine Learning}

Advanced AI and ML techniques form the intelligent core of the platform:

\begin{itemize}
    \item \textbf{Supervised Learning:} Random Forest for email classification
    \item \textbf{Unsupervised Learning:} Isolation Forest for anomaly detection
    \item \textbf{Natural Language Processing:} TF-IDF vectorization, BERT embeddings
    \item \textbf{Explainable AI:} LIME for model interpretability
    \item \textbf{Ensemble Methods:} Weighted voting for improved accuracy
\end{itemize}

\subsection{Cybersecurity and Information Assurance}

Security domain expertise is essential for effective threat detection:

\begin{itemize}
    \item \textbf{Threat Intelligence:} Understanding of phishing tactics and attack patterns
    \item \textbf{Security Operations:} Alignment with SOC workflows and SIEM integration
    \item \textbf{Incident Response:} Actionable recommendations and alerting
    \item \textbf{Compliance:} GDPR, KVKK, and industry regulations
    \item \textbf{Risk Management:} Unified risk scoring methodology
\end{itemize}

\subsection{Human-Computer Interaction}

Effective security tools must be usable by human operators:

\begin{itemize}
    \item \textbf{Dashboard Design:} Intuitive visualization of complex threat data
    \item \textbf{Explainability:} Presenting AI decisions in human-understandable terms
    \item \textbf{User Experience:} Streamlined workflows for security analysts
    \item \textbf{Accessibility:} Web-based interface accessible from any device
\end{itemize}

\subsection{Statistics and Data Science}

Statistical foundations underpin the ML components:

\begin{itemize}
    \item \textbf{Feature Engineering:} Statistical analysis for feature selection
    \item \textbf{Model Evaluation:} Precision, recall, F1-score, ROC-AUC metrics
    \item \textbf{Hypothesis Testing:} Validation of model performance claims
    \item \textbf{Data Preprocessing:} Normalization, outlier handling
\end{itemize}

\subsection{Project Management}

Successful delivery required project management skills:

\begin{itemize}
    \item \textbf{Agile Methodology:} Iterative development with regular milestones
    \item \textbf{Risk Management:} Identification and mitigation of project risks
    \item \textbf{Documentation:} Comprehensive technical and user documentation
    \item \textbf{Quality Assurance:} Testing strategy and validation
\end{itemize}

\subsection{Summary of Interdisciplinary Contributions}

\begin{table}[H]
\centering
\begin{tabularx}{\textwidth}{|l|X|}
\hline
\textbf{Discipline} & \textbf{Key Contributions to Project} \\
\hline
Computer Science & Core algorithms, data structures, system architecture \\
\hline
Machine Learning & Threat detection models, ensemble methods \\
\hline
NLP & Text analysis for email classification \\
\hline
Cybersecurity & Threat intelligence, attack pattern recognition \\
\hline
HCI & Dashboard design, explainable interfaces \\
\hline
Statistics & Model evaluation, feature engineering \\
\hline
Project Management & Planning, risk management, delivery \\
\hline
\end{tabularx}
\caption{Interdisciplinary Contributions Summary}
\end{table}

\newpage
