% ====== REFERENCES ======
\section{References}
\label{sec:references}

\begin{thebibliography}{99}

\bibitem{slashnext2024} SlashNext. (2024). \textit{2024 Phishing Intelligence Report}. SlashNext Security.

\bibitem{ribeiro2016} Ribeiro, M. T., Singh, S., \& Guestrin, C. (2016). ``Why Should I Trust You?'': Explaining the Predictions of Any Classifier. \textit{Proceedings of the 22nd ACM SIGKDD International Conference on Knowledge Discovery and Data Mining}, 1135--1144.

\bibitem{bergholz2010} Bergholz, A., De Beer, J., Glahn, S., Moens, M. F., Paaß, G., \& Strobel, S. (2010). New filtering approaches for phishing email. \textit{Journal of Computer Security}, 18(1), 7--35.

\bibitem{abunimeh2007} Abu-Nimeh, S., Nappa, D., Wang, X., \& Nair, S. (2007). A comparison of machine learning techniques for phishing detection. \textit{Proceedings of the Anti-Phishing Working Groups 2nd Annual eCrime Researchers Summit}, 60--69.

\bibitem{fette2007} Fette, I., Sadeh, N., \& Tomasic, A. (2007). Learning to detect phishing emails. \textit{Proceedings of the 16th International Conference on World Wide Web}, 649--656.

\bibitem{hiransha2018} Hiransha, M., Gopalakrishnan, E. A., Menon, V. K., \& Soman, K. P. (2018). Deep learning based phishing detection system. \textit{Procedia Computer Science}, 132, 1351--1361.

\bibitem{ma2009} Ma, J., Saul, L. K., Savage, S., \& Voelker, G. M. (2009). Beyond blacklists: learning to detect malicious web sites from suspicious URLs. \textit{Proceedings of the 15th ACM SIGKDD International Conference on Knowledge Discovery and Data Mining}, 1245--1254.

\bibitem{roesch1999} Roesch, M. (1999). Snort: Lightweight intrusion detection for networks. \textit{Proceedings of LISA '99: 13th Systems Administration Conference}, 229--238.

\bibitem{lane1999} Lane, T., \& Brodley, C. E. (1999). Temporal sequence learning and data reduction for anomaly detection. \textit{ACM Transactions on Information and System Security (TISSEC)}, 2(3), 295--331.

\bibitem{kruegel2003} Kruegel, C., \& Vigna, G. (2003). Anomaly detection of web-based attacks. \textit{Proceedings of the 10th ACM Conference on Computer and Communications Security}, 251--261.

\bibitem{liu2008} Liu, F. T., Ting, K. M., \& Zhou, Z. H. (2008). Isolation forest. \textit{2008 Eighth IEEE International Conference on Data Mining}, 413--422.

\bibitem{zhang2019} Zhang, Y., Li, P., \& Wang, X. (2019). Intrusion detection for IoT based on improved random forest algorithm. \textit{Wireless Communications and Mobile Computing}, 2019.

\bibitem{alshammari2009} Alshammari, R., \& Zincir-Heywood, A. N. (2009). Machine learning based encrypted traffic classification. \textit{Proceedings of the IEEE Symposium on Computational Intelligence in Security and Defense Applications}, 1--8.

\bibitem{lundberg2017} Lundberg, S. M., \& Lee, S. I. (2017). A unified approach to interpreting model predictions. \textit{Advances in Neural Information Processing Systems}, 30, 4765--4774.

\bibitem{marino2018} Marino, D. L., Wickramasinghe, C. S., \& Manic, M. (2018). An adversarial approach for explainable AI in intrusion detection systems. \textit{IECON 2018 -- 44th Annual Conference of the IEEE Industrial Electronics Society}, 3237--3243.

\bibitem{slack2020} Slack, D., Hilgard, S., Jia, E., Singh, S., \& Lakkaraju, H. (2020). Fooling LIME and SHAP: Adversarial attacks on post hoc explanation methods. \textit{AIES 2020 -- Proceedings of the AAAI/ACM Conference on AI, Ethics, and Society}, 180--186.

\bibitem{kuppa2020} Kuppa, A., \& Le-Khac, N. A. (2020). Black box attacks on explainable artificial intelligence (XAI) methods in cyber security. \textit{2020 International Joint Conference on Neural Networks}, 1--8.

\bibitem{bhatt2020} Bhatt, U., Xiang, A., Sharma, S., Weller, A., Taly, A., Jia, Y., Ghosh, J., Puri, R., Moura, J. M., \& Eckersley, P. (2020). Explainable machine learning in deployment. \textit{Proceedings of the 2020 Conference on Fairness, Accountability, and Transparency}, 648--657.

\bibitem{sadighian2013} Sadighian, A., Fernandez, J. M., Lemay, A., \& Zargar, S. T. (2013). ONTIDS: A highly flexible context-aware and ontology-based alert correlation framework. \textit{International Symposium on Foundations and Practice of Security}, 161--177.

\bibitem{friedberg2015} Friedberg, I., Skopik, F., Settanni, G., \& Fiedler, R. (2015). Combating advanced persistent threats: From network event correlation to incident detection. \textit{Computers \& Security}, 48, 35--57.

\bibitem{ghafir2018} Ghafir, I., Hammoudeh, M., Prenosil, V., Han, L., Heesook, R., Khanam, S., Rabie, S., \& Jazani, K. (2018). Detection of advanced persistent threats using machine-learning correlation analysis. \textit{Future Generation Computer Systems}, 89, 349--359.

\bibitem{shukla2020} Shukla, M., Verma, A., Agarwal, G., \& Rawat, S. (2020). A novel approach for network intrusion detection using multistage deep learning image recognition. \textit{Cluster Computing}, 23(4), 3105--3114.

\bibitem{devlin2018} Devlin, J., Chang, M. W., Lee, K., \& Toutanova, K. (2018). BERT: Pre-training of deep bidirectional transformers for language understanding. \textit{arXiv preprint arXiv:1810.04805}.

\bibitem{bojanowski2017} Bojanowski, P., Grave, E., Joulin, A., \& Mikolov, T. (2017). Enriching word vectors with subword information. \textit{Transactions of the Association for Computational Linguistics}, 5, 135--146.

\bibitem{flask2010} Ronacher, A. (2010). Flask: A Python Microframework. \url{https://flask.palletsprojects.com/}

\bibitem{scikit2011} Pedregosa, F., et al. (2011). Scikit-learn: Machine Learning in Python. \textit{Journal of Machine Learning Research}, 12, 2825--2830.

\bibitem{iso27001} ISO/IEC 27001:2013. \textit{Information technology -- Security techniques -- Information security management systems -- Requirements}.

\bibitem{nist_csf} National Institute of Standards and Technology. (2018). \textit{Framework for Improving Critical Infrastructure Cybersecurity, Version 1.1}.

\bibitem{gdpr} General Data Protection Regulation (GDPR). (2016). Regulation (EU) 2016/679 of the European Parliament and of the Council.

\bibitem{mitre_attack} MITRE Corporation. (2020). \textit{MITRE ATT\&CK Framework}. \url{https://attack.mitre.org/}

\end{thebibliography}

\newpage
