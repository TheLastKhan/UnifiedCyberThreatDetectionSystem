% ====== RISK MANAGEMENT ======
\section{Risk Management}

Effective risk management is essential for the successful completion of this project. The following section identifies potential risks, assesses their probability and impact, and outlines mitigation strategies.

\subsection{Risk Assessment Matrix}

\begin{longtable}{|p{0.5cm}|p{3.5cm}|p{1.5cm}|p{1.5cm}|p{1.5cm}|p{5cm}|}
\hline
\textbf{ID} & \textbf{Risk Description} & \textbf{Prob.} & \textbf{Impact} & \textbf{Score} & \textbf{Mitigation Strategy} \\
\hline
\endhead

R1 & Insufficient training data quality & Medium & High & H & Collect data from multiple sources (Kaggle, synthetic generation). Implement data augmentation. \\
\hline

R2 & Model performance below target & Medium & High & H & Hyperparameter tuning pipeline. Alternative algorithms ready (SVM, XGBoost). Ensemble methods. \\
\hline

R3 & Integration complexity between modules & High & Medium & H & Modular architecture design. Clear API contracts. Incremental integration testing. \\
\hline

R4 & Explainability computation overhead & Medium & Medium & M & Caching for explanations. Lazy computation. On-demand explanation generation. \\
\hline

R5 & Security vulnerabilities in web app & Medium & High & H & OWASP guidelines adherence. Input validation. Security testing. Code review. \\
\hline

R6 & Performance bottlenecks in real-time analysis & Medium & High & H & Profiling and optimization. Async processing. Redis caching layer. \\
\hline

R7 & Scope creep & High & Medium & H & Clear requirements documentation. Change control process. Prioritized feature backlog. \\
\hline

R8 & Team member unavailability & Low & High & M & Documentation for knowledge transfer. Modular task assignment. \\
\hline

R9 & Third-party library compatibility & Medium & Medium & M & Dependency version pinning. Virtual environments. Regular dependency updates. \\
\hline

R10 & Concept drift in prod deployment & Medium & Medium & M & Model monitoring infrastructure. Retraining pipeline. Performance alerts. \\
\hline

\end{longtable}

\subsection{Risk Scoring Methodology}

Risk scores are calculated as:
\begin{equation}
\text{Risk Score} = \text{Probability} \times \text{Impact}
\end{equation}

\begin{table}[H]
\centering
\begin{tabular}{|c|c|c|c|c|}
\hline
& \multicolumn{4}{c|}{\textbf{Impact}} \\
\cline{2-5}
\textbf{Probability} & Low & Medium & High & Critical \\
\hline
High & M & H & H & C \\
Medium & L & M & H & H \\
Low & L & L & M & H \\
\hline
\end{tabular}
\caption{Risk Matrix (L=Low, M=Medium, H=High, C=Critical)}
\end{table}

\subsection{Monitored Risks and Status}

\begin{table}[H]
\centering
\begin{tabular}{|c|l|c|c|}
\hline
\textbf{Risk ID} & \textbf{Current Status} & \textbf{Action Taken} & \textbf{Outcome} \\
\hline
R1 & Mitigated & Multiple datasets collected & 100K+ labeled emails \\
R2 & Mitigated & RF achieved 89\% accuracy & Target met \\
R3 & Ongoing & API contracts defined & Integration smooth \\
R4 & Mitigated & Caching implemented & $<$2s response \\
R5 & Ongoing & Security review planned & Tests passing \\
\hline
\end{tabular}
\caption{Risk Monitoring Summary}
\end{table}

\subsection{Contingency Planning}

For high-impact risks, the following contingency plans are in place:

\textbf{Model Performance Failure (R2):}
\begin{itemize}
    \item Fallback to simpler rule-based classification
    \item Pre-trained BERT model as backup
    \item Ensemble voting to improve accuracy
\end{itemize}

\textbf{Security Vulnerability Discovery (R5):}
\begin{itemize}
    \item Immediate patch deployment process
    \item Incident response plan documented
    \item User notification protocol
\end{itemize}

\textbf{System Performance Issues (R6):}
\begin{itemize}
    \item Load balancing across workers
    \item Feature reduction for real-time path
    \item Batch processing option for complex analysis
\end{itemize}

\newpage
